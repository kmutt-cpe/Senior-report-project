%%%%%% Run at command line, run
%%%%%% xelatex grad-sample.tex 
%%%%%% for a few times to generate the output pdf file
\documentclass[12pt,oneside,openright,a4paper]{cpe-english-project}

\usepackage{polyglossia}
\setdefaultlanguage{english}
\setotherlanguage{thai}
\newfontfamily\thaifont[Script=Thai,Scale=1.23]{TH Sarabun New}
\defaultfontfeatures{Mapping=tex-text,Scale=1.0,LetterSpace=0.0}
\setmainfont[Scale=1.0,LetterSpace=0,WordSpace=1.0,FakeStretch=1.0]{Times New Roman}
%\setmathfont(Digits)[Scale=1.0,LetterSpace=0,FakeStretch=1.0]{Times New Roman}


%%%%%%%%%%%%%%%%%%%%%%%%%%%%%%%%%%%%%%%%%%%%%%%%%%%%%%%%%%%%%%%%%%%
% Customize below to suit your needs 
% The ones that are optional can be left blank. 
%%%%%%%%%%%%%%%%%%%%%%%%%%%%%%%%%%%%%%%%%%%%%%%%%%%%%%%%%%%%%%%%%%%
% First line of title
\def\disstitleone{KMUTT CPE Chatbot System}   
% Second line of title
\def\disstitletwo{}   
% Your first name and lastname
\def\dissauthor{Mr. Nathaphop Sundarabhogin}   % 1st member
%%% Put other group member names here ..
\def\dissauthortwo{Ms. Natkanok Poksappaiboon}   % 2nd member (optional)
\def\dissauthorthree{Mr. Natthawat Tungruethaipak}   % 3rd member (optional)


% The degree that you're persuing..
\def\dissdegree{Bachelor of Engineering} % Name of the degree
\def\dissdegreeabrev{B.Eng} % Abbreviation of the degree
\def\dissyear{2020}                   % Year of submission
\def\thaidissyear{2563}               % Year of submission (B.E.)

%%%%%%%%%%%%%%%%%%%%%%%%%%%%%%%%%%%%%%%%%%%%
% Your project and independent study committee..
%%%%%%%%%%%%%%%%%%%%%%%%%%%%%%%%%%%%%%%%%%%%
\def\dissadvisor{Asst. Prof. Santitham Prom-on, Ph.D.}  % Advisor
%%% Leave it empty if you have no Co-advisor
\def\disscoadvisor{}  % Co-advisor
\def\disscommitteetwo{Assoc. Prof. Dr. Peerapon Siripongwutikorn, Ph.D.}  % 3rd committee member (optional)
\def\disscommitteethree{Asst. Prof. Nuttanart Facundes, Ph.D.}   % 4th committee member (optional) 
\def\disscommitteefour{Assoc. Prof. Thumrongrat Amornraksa, Ph.D.}    % 5th committee member (optional) 

\def\worktype{Project} %%  Project or Independent study
\def\disscredit{3}   %% 3 credits or 6 credits


\def\fieldofstudy{Computer Engineering} 
\def\department{Computer Engineering} 
\def\faculty{Engineering}

\def\thaifieldofstudy{วิศวกรรมคอมพิวเตอร์} 
\def\thaidepartment{วิศวกรรมคอมพิวเตอร์} 
\def\thaifaculty{วิศวกรรมศาสตร์}
 
\def\appendixnames{Appendix} % todo: Select Appendices or Appendix

\def\thaiworktype{ปริญญานิพนธ์} %  Project or research project % 
\def\thaidisstitleone{ระบบแชทบอทภาควิชาวิศวกรรมคอมพิวเตอร์ มจธ.}
\def\thaidisstitletwo{}
\def\thaidissauthor{นายณฐาภพ สุนทรโภคิน}
\def\thaidissauthortwo{นางสาวณัฐกนก โภคทรัพย์ไพบูลย์} %Optional
\def\thaidissauthorthree{นายณัฐวัฒน์ ตั้งฤทัยภักดิ์} %Optional

\def\thaidissadvisor{ผศ.ดร. สันติธรรม พรหมอ่อน}
%% Leave this empty if you have no co-advisor
\def\thaidisscoadvisor{} %Optional
\def\thaidissdegree{วิศวกรรมศาสตรบัณฑิต}

% Change the line spacing here...
\linespread{1.15}

%%%%%%%%%%%%%%%%%%%%%%%%%%%%%%%%%%%%%%%%%%%%%%%%%%%%%%%%%%%%%%%%
% End of personal customization.  Do not modify from this part 
% to \begin{document} unless you know what you are doing...
%%%%%%%%%%%%%%%%%%%%%%%%%%%%%%%%%%%%%%%%%%%%%%%%%%%%%%%%%%%%%%%%


%%%%%%%%%%%% Dissertation style %%%%%%%%%%%
%\linespread{1.6} % Double-spaced  
%%\oddsidemargin    0.5in
%%\evensidemargin   0.5in
%%%%%%%%%%%%%%%%%%%%%%%%%%%%%%%%%%%%%%%%%%%
%\renewcommand{\subfigtopskip}{10pt}
%\renewcommand{\subfigbottomskip}{-5pt} 
%\renewcommand{\subfigcapskip}{-6pt} %vertical space between caption
%                                    %and figure.
%\renewcommand{\subfigcapmargin}{0pt}

\renewcommand{\topfraction}{0.85}
\renewcommand{\textfraction}{0.1}

\newtheorem{theorem}{Theorem}
\newtheorem{lemma}{Lemma}
\newtheorem{corollary}{Corollary}

\def\QED{\mbox{\rule[0pt]{1.5ex}{1.5ex}}}
\def\proof{\noindent\hspace{2em}{\itshape Proof: }}
\def\endproof{\hspace*{\fill}~\QED\par\endtrivlist\unskip}
%\newenvironment{proof}{{\sc Proof:}}{~\hfill \blacksquare}
%% The hyperref package redefines the \appendix. This one 
%% is from the dissertation.cls
%\def\appendix#1{\iffirstappendix \appendixcover \firstappendixfalse \fi \chapter{#1}}
%\renewcommand{\arraystretch}{0.8}
%%%%%%%%%%%%%%%%%%%%%%%%%%%%%%%%%%%%%%%%%%%%%%%%%%%%%%%%%%%%%%%%
%%%%%%%%%%%%%%%%%%%%%%%%%%%%%%%%%%%%%%%%%%%%%%%%%%%%%%%%%%%%%%%%


\begin{document}
\pdfstringdefDisableCommands{%
\let\MakeUppercase\relax
}
\begin{center}
  \includegraphics[width=2.8cm]{./img/cover/logo02.jpg}
\end{center}
\vspace*{-1cm}

\maketitlepage
\makesignaturepage 

%%%%%%%%%%%%%%%%%%%%%%%%%%%%%%%%%%%%%%%%%%%%%%%%%%%%%%%%%%%%%%
%%%%%%%%%%%%%%%%%%%%%% English abstract %%%%%%%%%%%%%%%%%%%%%%%
%%%%%%%%%%%%%%%%%%%%%%%%%%%%%%%%%%%%%%%%%%%%%%%%%%%%%%%%%%%%%%
\abstract

Our world is becoming more technology driven. The population needs to have an awareness of technology 
like Artificial Intelligence (AI) for the country to become successful and reduce the workload of humans, as 
in the Computer Engineering department of King Mongkut’s University of Technology Thonburi. \\
Nowadays, many outsiders and insiders, whether students, teacher assistants, and staff have asked many 
questions about departments, such as frequently asked questions (FAQ) or the curriculum of each year etc. 
Each day, the staff in the department have to answer the same questions repeatedly. \\
Therefore, our team wants to develop the KMUTT CPE Chatbot System to help in answering the frequently 
asked questions and general inquiries of the Computer Engineering department in KMUTT to outsiders. It 
can also bring questions that chatbot cannot reply to officers and administration for answering the question 
and then the chatbot will send the answer back to the questioner via the Facebook Messenger platform.

\begin{flushleft}
\begin{tabular*}{\textwidth}{@{}lp{0.8\textwidth}}
\textbf{Keywords}: & Knowledge management / Chatbot / Artificial Intelligence (AI) / Siamese Neural Network / MySQL / Database / Automation / FAQ
\end{tabular*}
\end{flushleft}
\endabstract

%%%%%%%%%%%%%%%%%%%%%%%%%%%%%%%%%%%%%%%%%%%%%%%%%%%%%%%%%%%%%%
%%%%%%%%%% Thai abstract here %%%%%%%%%%%%%%%%%%%%%%%%%%%%%%%%%
%%%%%%%%%%%%%%%%%%%%%%%%%%%%%%%%%%%%%%%%%%%%%%%%%%%%%%%%%%%%%%
{
\XeTeXlinebreaklocale "th_TH"	
\thaifont
\thaiabstract
ในโลกของเราได้มีการขับเคลื่อนด้วยเทคโนโลยีที่มากขึ้น ดังนั้นประชากรจึงต้องมีการปรับตัวและรู้ทันถึงเทคโนโลยี เช่น ปัญญาประดิษฐ์ หรือ 
เอไอ เพื่อที่ประเทศนั้นจะสามารถพัฒนาและประสบความสำเร็จในการลดการใช้แรงงานของบุคลากรลง อนึ่งเช่นบุคลากรของภาควิชาวิศวกรรม
คอมพิวเตอร์ มหาวิทยาลัยเทคโนโลยีพระจอมเกล้าธนบุรี\\
ปัจจุบัน ทั้งคนภายนอกและภายใน ไม่แม้แต่นักศึกษา คุณครู ผู้ช่วยสอน หรือบุคลากรนั้น ล้วนแล้วแต่มีคำถามเกี่ยวกับภาควิชาวิศวกรรม
คอมพิวเตอร์ เช่น หลักสูตร การเรียนการสอนแต่ละปี ฯลฯ จนเกิดเป็นคำถามที่สามารถพบเห็นได้บ่อย ทำให้ภาควิชานั้นต้องมีการตอบ
คำถามที่บ่อยครั้งและเดิม ๆ\\
ดังนั้นทางผู้จัดทำจึงมีการเล็งเห็นปัญหาและต้องการที่จะทำโปรแกรมแชทบอทของภาควิชาวิศวกรรมคอมพิวเตอร์ มหาวิทยาลัยเทคโนโลยี
พระจอมเกล้าธนบุรีขึ้นมา เพื่อเป็นการช่วยตอบคำถามที่พบเห็นได้บ่อยและข้อสงสัยต่าง ๆ เกี่ยวกับภาควิชาแก่บุคคลภายนอกได้ โดยโปรแกรม
นั้นยังสามารถนำเอาคำถามที่แชทบอทไม่สามารถตอบได้นั้นส่งต่อไปให้แก่บุคลากรหรือเจ้าหน้าที่ดูแลเป็นผู้ตอบคำถามเหล่านั้นโดยผ่าน
\\แอปพลิเคชัน Facebook Messenger

\begin{flushleft}
\begin{tabular*}{\textwidth}{@{}lp{0.8\textwidth}}
 & \\

\textbf{คำสำคัญ}: & การจัดการความรู้ / Chatbot / ปัญญาประดิษฐ์ / Siamese Neural Network / MySQL / ฐานข้อมูล / Automation / คำถามที่พบบ่อย
\end{tabular*}
\end{flushleft}
\endabstract
}

%%%%%%%%%%%%%%%%%%%%%%%%%%%%%%%%%%%%%%%%%%%%%%%%%%%%%%%%%%%%
%%%%%%%%%%%%%%%%%%%%%%% Acknowledgments %%%%%%%%%%%%%%%%%%%%
%%%%%%%%%%%%%%%%%%%%%%%%%%%%%%%%%%%%%%%%%%%%%%%%%%%%%%%%%%%%
\preface
First, we would like to express our gratitude toward our advisor Asst. Prof. Santitham Prom-on whom always 
there to help us. When we need some advice, had a question about our research or a problem in writing, he 
always supports us even in times that he is busy, he still finds some time or alternative way to help us do our 
research. We are very grateful for what you have done for us, thank you.
\\Secondly, we owe our deep gratitude to all our committees: Assoc. Prof. Peerapon Siripongwutikorn, Asst. 
Prof. Nuttanart Facundes, and Assoc. Prof. Thumrongrat Amornraksa, who interest in our project and guided 
us all along until the completion.  By providing all the necessary information for developing a good system, 
give us advice and comment to make our project and ourselves to become better.
\\Last, we would like to thank our friends and family who, support us and help us in many ways besides the 
project, such as encouragement, guideline, and many other ways. We are very grateful toward all of them, 
thank you.


%%%%%%%%%%%%%%%%%%%%%%%%%%%%%%%%%%%%%%%%%%%%%%%%%%%%%%%%%%%%%
%%%%%%%%%%%%%%%% ToC, List of figures/tables %%%%%%%%%%%%%%%%
%%%%%%%%%%%%%%%%%%%%%%%%%%%%%%%%%%%%%%%%%%%%%%%%%%%%%%%%%%%%%
% The three commands below automatically generate the table 
% of content, list of tables and list of figures
\tableofcontents                    
\listoftables
\listoffigures                      

%%%%%%%%%%%%%%%%%%%%%%%%%%%%%%%%%%%%%%%%%%%%%%%%%%%%%%%%%%%%%%
%%%%%%%%%%%%%%%%%%%%% List of symbols page %%%%%%%%%%%%%%%%%%%
%%%%%%%%%%%%%%%%%%%%%%%%%%%%%%%%%%%%%%%%%%%%%%%%%%%%%%%%%%%%%%
% You have to add this manually..
\listofsymbols
\begin{flushleft}
\begin{tabular}{@{}p{0.07\textwidth}p{0.7\textwidth}p{0.1\textwidth}}
\textbf{SYMBOL}  & & \textbf{UNIT} \\[0.2cm]
$\alpha$ & Test variable\hfill & m$^2$ \\
$\lambda$ & Interarival rate\hfill &  jobs/second\\
$\mu$ & Service rate\hfill & jobs/second\\
\end{tabular}
\end{flushleft}
%%%%%%%%%%%%%%%%%%%%%%%%%%%%%%%%%%%%%%%%%%%%%%%%%%%%%%%%%%%%%%
%%%%%%%%%%%%%%%%%%%%% List of vocabs & terms %%%%%%%%%%%%%%%%%
%%%%%%%%%%%%%%%%%%%%%%%%%%%%%%%%%%%%%%%%%%%%%%%%%%%%%%%%%%%%%%
% You also have to add this manually..
\listofvocab
\begin{flushleft}
\begin{tabular}{@{}p{1in}@{=\extracolsep{0.5in}}l}
ABC & Adaptive Bandwidth Control \\
MANET & Mobile Ad Hoc Network 
\end{tabular}
\end{flushleft}

%\setlength{\parskip}{1.2mm}

%%%%%%%%%%%%%%%%%%%%%%%%%%%%%%%%%%%%%%%%%%%%%%%%%%%%%%%%%%%%%%%
%%%%%%%%%%%%%%%%%%%%%%%% Main body %%%%%%%%%%%%%%%%%%%%%%%%%%%%
%%%%%%%%%%%%%%%%%%%%%%%%%%%%%%%%%%%%%%%%%%%%%%%%%%%%%%%%%%%%%%%


\chapter{Introduction}

\section{Background} 

Explain the background of your works for readers. You can refer to figure by like this.. Figure~\ref{fig:x1}.
\begin{figure}[!h]
\caption{This is the figure x11}\label{fig:x1}
\end{figure}

Get ready, skanks! It's time for the truth train! Fame was like a drug. But what was even more like a drug were the drugs. Attempted murder? Now honestly, what is that? Do they give a Nobel Prize for attempted chemistry?

"Thank the Lord"? That sounded like a prayer. A prayer in a public school. God has no place within these walls, just like facts don't have a place within an organized religion. Thank you, steal again. I hope this has taught you kids a lesson: kids never learn.

…And the fluffy kitten played with that ball of string all through the night. On a lighter note, a Kwik-E-Mart clerk was brutally murdered last night. Oh, so they have Internet on computers now!

You don't like your job, you don't strike. You go in every day and do it really half-assed. That's the American way. Lisa, vampires are make-believe, like elves, gremlins, and Eskimos. Jesus must be spinning in his grave!

I prefer a vehicle that doesn't hurt Mother Earth. It's a go-cart, powered by my own sense of self-satisfaction. Marge, it takes two to lie. One to lie and one to listen. Attempted murder? Now honestly, what is that? Do they give a Nobel Prize for attempted chemistry?

I was saying "Boo-urns." Bart, with \$10,000 we'd be millionaires! We could buy all kinds of useful things like…love! I'll keep it short and sweet — Family. Religion. Friendship. These are the three demons you must slay if you wish to succeed in business.

How could you?! Haven't you learned anything from that guy who gives those sermons at church? Captain Whatshisname? We live in a society of laws! Why do you think I took you to all those Police Academy movies? For fun? Well, I didn't hear anybody laughing, did you? Except at that guy who made sound effects. Makes sound effects and laughs. Where was I? Oh yeah! Stay out of my booze. I can't go to juvie. They use guys like me as currency.

Oh, loneliness and cheeseburgers are a dangerous mix. "Thank the Lord"? That sounded like a prayer. A prayer in a public school. God has no place within these walls, just like facts don't have a place within an organized religion.


\section{Motivations}
Explain the motivations of your works.  
\begin{itemize}
\item   What are the problems you are addressing? 
\item  Why they are important?
\item  What are the limitations of existing approaches? 
\end{itemize}
You may combine this section with the background section.


\section{Problem Statements}

\section{Objectives}


\section{Scope of Work}

Explain the scope of your works. 
\begin{itemize}
\item   What are the problems you are addressing? 
\item  Why they are important?
\item  What are the limitations of existing approaches?
\end{itemize}

\section{Project Schedule}


%%%%%%%%%%%%%%%%%%%%%%%%%%%%%%%%%%%%%%%%%%%%%%%%%%%%%%%%%%%%
%%%%%%%%%%%%%%  Literature Review %%%%%%%%%%%%%%%%%%%%%%%%%%
%%%%%%%%%%%%%%%%%%%%%%%%%%%%%%%%%%%%%%%%%%%%%%%%%%%%%%%%%%%%
\chapter{Background Theory and Related Work}

\section{Introduction}
This chapter, we are going to demonstrate the theory, core concept, tools, and related research to be background 
and terminology before deep down into methodology and result.

\section{Theory and Core Concepts}
\subsection{Knowledge Management}
Knowledge Management is the process of creating, sharing, gathering, using and managing the knowledge 
and information in organization (Koenig, 2018). It refers to a multidisciplinary approach to achieve organizational 
objectives by making the best use of knowledge.\\
In terms of our project, knowledge management is the website to keep the knowledge and information about 
computer engineering in question and answer format. Questions in knowledge management can group in category 
such as basic information, curriculum, frequently asked questions (FAQ), and registration. Also, in each category 
has subcategory to be more specific the question. Only staff in Computer Engineering department can use the knowledge 
management.

\begin{table}[!h]
\caption{test table method1}\label{tbl:method1}
\begin{tabular}{c|c|l|rr} \hline\hline
Center & Center & left aligned & Right & Right aligned \\ \hline\hline
Center & Center & left aligned & Right & Right aligned \\ \hline
Center & Center & left aligned & Right & Right aligned \\ 
Center & Center & left aligned & Right & Right aligned \\ \hline
Center & Center & left aligned & Right & Right aligned \\ \hline\hline
\end{tabular}
\end{table}


\section{Text Processing Algorithms}
\subsection{Algorithm I}

% Can define this in the preamble..
You can place the figure and refer to it as Figure~\ref{fig:model2}.
The figure and table numbering will be run and updated automatically when you add/remove tables/figures from the document.

\begin{figure}[!h]\centering
\setlength{\fboxrule}{0.2mm} % can define this in the preamble
\setlength{\fboxsep}{1cm}
\fbox{\includegraphics[width=5cm]{./model2.pdf}}
\caption{The network model}\label{fig:model2}
\end{figure}

 
\subsection{Algorithm II}
Add more subsections as you want.


\section{Development Tools}

%%%%%%%%%%%%%%%%%%%%%%%%%%%%%%%%%%%%%%%%%%%%%%%%%%%%%55
%%%%%%%%%%%%%%%%%%%%%%%%%%%%%%%%%%%%%%%%%%%%%%%%%%%%%
%%%%%%%%%%%%%%%%%%%%%%%%%%%%%%%%%%%%%%%%%%%%%%%%%%%%%
\chapter{Proposed Work}

Explain the design (how you plan to implement your work) of your project. Adjust the section titles below to suit the types of your work. Detailed physical design like circuits and source codes should be placed in the appendix.

\section{System Architecture}

\begin{table}[!h]
\centering
\caption{test table x1}\label{tbl:symbols}
\begin{tabular}{@{}p{0.07\textwidth}|p{0.7\textwidth}p{0.1\textwidth}}\hline
\multicolumn{2}{l}{\textbf{SYMBOL}}  & \textbf{UNIT} \\ \hline 
$\alpha$ & Test variable\hfill & m$^2$ \\
$\lambda$ & Interarrival rate\hfill &  jobs/second\\
$\mu$ & Service rate\hfill & jobs/second \\ \hline
\end{tabular}
%\begin{tabular}{c|c} \hline
% $\alpha$ & $\beta$ \\ \hline
% $\delta$ & $\mu$ \\ \hline
%\end{tabular}
\end{table}

\section{System Specifications and Requirements}

\section{Hardware Module 1}
\subsection{Component 1}
\subsection{Logical Circuit Diagram}

\section{Hardware Module 2}
\subsection{Component 1}
\subsection{Component 2}

\section{Path Finding Algorithm}

\section{Database Design}

\section{GUI Design}



%%%%%%%%%%%%%%%%%%%%%%%%%%%%%%%%%%%%%%%%%%%%%%%%%%%%%%%%%%%%%%
%%%%%%%%%%%%%%%%%%%% Experiments %%%%%%%%%%%%%%%%%%%%%%%%%%%%%
%%%%%%%%%%%%%%%%%%%%%%%%%%%%%%%%%%%%%%%%%%%%%%%%%%%%%%%%%%%%%%%
\chapter{Implementation Results}
\section{Chatbot Subsystem and Storage Subsystem Result}
Now our chatbot subsystem which we implement in dialog flow can greeting and understand what is the question and for storage subsystem which we implement in GraphQL query category from backend API. Now our system can category of FAQs by ID and can query FAQs from all category.

\begin{figure}[!h]\centering
\fbox{\includegraphics[width=15cm]{./img/ch4/ch4_1.png}}
\caption{Classification greeting messages and questions}\label{fig:Classification greeting messages and questions}
\end{figure}
\begin{figure}[!h]\centering
\fbox{\includegraphics[width= 15cm]{./img/ch4/ch4_2.png}}
\caption{GraphQL query FAQs category from backend API}\label{fig:GraphQL query FAQs category from backend API}
\end{figure}

\section{Model Prediction Subsystem Result}
\subsection{Training Siamese Neural Network with MaLSTM Result}
After preprocessing the data, we split the data into train, validate, and test data. Then separate the data into left inputs and right inputs, one for each side of MaLSTM. Then train the model with 100 epochs and use batch sizes 128. In addition to using early stops to prevent the model too overfitting and model checkpoint to save the model. Even our group set 100 epochs, but the model was trained only 17 epochs before stopping from early stop. Loss of training data is 0.4116 and loss of validation data is 0.4358. Also, accuracy of training data is 0.5487 and accuracy of validation data is 0.4358. Graphs below are the result from training.

\begin{figure}[!h]\centering
\fbox{\includegraphics[width= 15cm]{./img/ch4/ch4_3.png}}
\caption{Model accuracy}\label{fig:Model accuracy}
\end{figure}
\begin{figure}[!h]\centering
\fbox{\includegraphics[width= 15cm]{./img/ch4/ch4_4.png}}
\caption{Model loss}\label{fig:Model loss}
\end{figure}

\subsection{Evaluation Model Result}
The result from the model is the real number from 0 to 1. We interpret both questions are similar (result = 1) if the model predicted score is more than 0.3. If the model predicted score is less than 0.3, both questions are not similar (result = 0). We pick 0.3 to be a criterion for similar or not similar because we have picked 0.6 and 0.5 already but accuracy is very low. After the model predicts the test data, we convert from a similar score (0 to 1)  to binary (0 or 1) instead before evaluating. As you can see in the training topic, we use accuracy and loss to be the main criteria to evaluate the model while training. But in this process we will also use accuracy, precision, and F1-score to evaluate the model. The picture below is the result after evaluating the model with test data.

\begin{figure}[!h]\centering
\fbox{\includegraphics[width= 15cm]{./img/ch4/ch4_5.png}}
\caption{Classification report}\label{fig:Classification report}
\end{figure}
\begin{figure}[!h]\centering
\fbox{\includegraphics[width= 15cm]{./img/ch4/ch4_6.png}}
\caption{Confusion matrix}\label{fig:Confusion matrix}
\end{figure}

The accuracy after evaluating with test data is 0.527. From the confusion matrix and classification report, the overall model performs accuracy with 52.7\%. There are two classes and we got class o with less precision around 25\%. but we have got class 1 with high precision with 76\%.
\subsection{Siamese Neural Network with MaLSTM Model Result}

The picture below shows an example result from playing with the model. The Input questions are
{
\XeTeXlinebreaklocale "th_TH"	
\thaifont 
'วิศวะคอมเรียนหลักสูตรอะไรบ้างคะ' }
and
{
\XeTeXlinebreaklocale "th_TH"	
\thaifont 
 ' วิศวะคอมมีหลักสูตรอะไรบ้าง'. 
}Both questions are similar and our model can predict both questions are similar for 83.33\%.

\begin{figure}[!h]\centering
\fbox{\includegraphics[width= 15cm]{./img/ch4/ch4_7.png}}
\caption{Model result by compare questions which similar}\label{fig:Model result by compare questions which similar}
\end{figure}

From others example from playing with model.The Input questions are
{
\XeTeXlinebreaklocale "th_TH"	
\thaifont 
 ‘ วิศวะคอมเรียนเกี่ยวกับอะไรบ้าง’ }and 
{
\XeTeXlinebreaklocale "th_TH"	
\thaifont ‘ วิศวะคอมมีเรียนหลักสูตรอะไรบ้าง’.}
 Both questions are not similar and our model can predict both questions are not similar for 57.14\%

\begin{figure}[!h]\centering
\fbox{\includegraphics[width= 15cm]{./img/ch4/ch4_8.png}}
\caption{Model result by compare questions which not similar}\label{fig:Model result by compare questions which not similar}
\end{figure}

Even the model can predict, but the overall performance of the model is quite not good. Because we think the model should return similar questions from MaLSTM more than 0.5 scores not 0.3 scores. We think that the problem comes from not enough data. If we can gather more dataset and have more cases about question pairs that are similar or not similar, especially not similar that we got less precision, we can increase our model's performance. Also we have to integrate all subsystem to be one system in the production environment.

%%%%%%%%%%%%%%%%%%%%%%%%%%%%%%%%%%%%%%%%%%%%%%%%%%%%%%%%%%%%%%%
%%%%%%%%%%%%%%%%%%%% Conclusions %%%%%%%%%%%%%%%%%%%%%%%%%%%%%
%%%%%%%%%%%%%%%%%%%%%%%%%%%%%%%%%%%%%%%%%%%%%%%%%%%%%%%%%%%%%%%
\chapter{Conclusions}

This chapter is optional for proposal and progress reports but 
is required for the final report.

\section{Problems and Solutions}
State your problems and how you fixed them.

\section{Future Works}
What could be done in the future to make your projects better.

%%%%%%%%%%%%%%%%%%%%%%%%%%%%%%%%%%%%%%%%%%%%%%%%%%%%%%%%%%%%%%%
%%%%%%%%%%%%%%%%%%%% Bibliography %%%%%%%%%%%%%%%%%%%%%%%%%%%%%
%%%%%%%%%%%%%%%%%%%%%%%%%%%%%%%%%%%%%%%%%%%%%%%%%%%%%%%%%%%%%%%

%%%% Comment this in your report to show only references you have
%%%% cited. Otherwise, all the references below will be shown.
\nocite{*}
%% Use the kmutt.bst for bibtex bibliography style 
%% You must have cpe.bib and string.bib in your current directory.
%% You may go to file .bbl to manually edit the bib items.
\bibliographystyle{kmutt}
\bibliography{string,cpe}

%%%%%%%%%%%%%%%%%%%%%%%%%%%%%%%%%%%%%%%%%%%%%%%%%%%%%%%%%%%%%%%
%%%%%%%%%%%%%%%%%%%%%%%% Appendix %%%%%%%%%%%%%%%%%%%%%%%%%%%%%
%%%%%%%%%%%%%%%%%%%%%%%%%%%%%%%%%%%%%%%%%%%%%%%%%%%%%%%%%%%%%%%
\appendix{First appendix title}
\noindent{\large\bf Put appropriate topic here} \\

This is where you put hardware circuit diagrams, detailed experimental data in tables or source codes, etc.. \\ \bigskip



This appendix describes two static allocation methods for fGn (or fBm)
traffic. Here, $\lambda$ and $C$ are respectively the traffic arrival
rate and the service rate per dimensionless time step. Their unit are
converted to a physical time unit by multiplying the step size
$\Delta$. For a fBm self-similar traffic source,
Norros~\cite{norros95} provides its EB as
\begin{equation}\label{eq:norros}
  C = \lambda + (\kappa(H)\sqrt{-2\ln\epsilon})^{1/H}a^{1/(2H)}x^{-(1-H)/H}\lambda^{1/(2H)}
\end{equation}
where $\kappa(H) = H^H(1-H)^{(1-H)}$. Simplicity in the calculation is
the attractive feature of (\ref{eq:norros}).

The MVA technique developed in~\cite{kim01} so far provides the most
accurate estimation of the loss probability compared to previous
bandwidth allocation techniques according to simulation results.
Consider a discrete-time queueing system with constant service rate
$C$ and input process $\lambda_n$ with $\mathbb{E}\{\lambda_n\} =
\lambda$ and $\mathrm{Var}\{\lambda_n\} = \sigma^2$.  Define $X_n \equiv
\sum_{k=1}^n \lambda_k - Cn$.  The loss probability due to the MVA
approach is given by
\begin{equation}\label{eq:loss_mva}
  \varepsilon \approx \alpha e^{-m_x/2}
\end{equation}
where
\begin{equation}\label{eq:mx}
m_x = \min_{n \geq 0} \frac{((C-\lambda)n + B)^2}{\mathrm{Var}\{X_n\}} =
\frac{((C-\lambda)n^\ast + B)^2}{\mathrm{Var}\{X_{n^{\ast}}\}}
\end{equation} 
and 
\begin{equation}\label{eq:alpha}
  \alpha =
  \frac{1}{\lambda\sqrt{2\pi\sigma^2}}\exp\left(\frac{(C-\lambda)^2}{2\sigma^2}\right)
  \int_C^\infty (r-C)\exp\left(\frac{(r-\lambda)^2}{2\sigma^2}\right)\, dr
\end{equation}
For a given $\varepsilon$, we numerically solve for $C$ that satisfies
(\ref{eq:loss_mva}). Any search algorithm can be used to do the task.
Here, the bisection method is used.  

Next, we show how $\mathrm{Var}\{X_n\}$ can be determined.  Let
$C_{\lambda}(l)$ be the autocovariance function of $\lambda_n$.  The
MVA technique basically approximates the input process $\lambda_n$
with a Gaussian process, which allows $\mathrm{Var}\{X_n\}$ to be
represented by the autocovariance function.  In particular, the
variance of $X_n$ can be expressed in terms of $C_{\lambda}(l)$ as
\begin{equation}
  \mathrm{Var}\{X_n\} = nC_{\lambda}(0) + 2\sum_{l=1}^{n-1} (n-l)C_{\lambda}(l)
\end{equation} 
Therefore, $C_{\lambda}(l)$ must be known in the MVA technique, either
by assuming specific traffic models or by off-line analysis in case of
traces.  In most practical situations, $C_{\lambda}(l)$ will not be
known in advance, and an on-line measurement algorithm developed
in~\cite{eun01} is required to jointly determine both $n^\ast$ and
$m_x$. For fGn traffic, $\mathrm{Var}\{X_n\}$ is equal to $\sigma^2
n^{2H}$, where $\sigma^2 = \mathrm{Var}\{\lambda_n\}$, and we can find
the $n^\ast$ that minimizes (\ref{eq:mx}) directly. Although $\lambda$
can be easily measured, it is not the case for $\sigma^2$ and $H$.
Consequently, the MVA technique suffers from the need of prior
knowledge traffic parameters.


%%%%%%%%%%%%%%%%%%%%%%%%%%%%%%%%%%%%%%%%%%%%%%%%%%%%%%%%%%
%%%%%%%%%%%%%%% The 2nd appendix %%%%%%%%%%%%%%%%%%%%%%%%%%
%%%%%%%%%%%%%%%%%%%%%%%%%%%%%%%%%%%%%%%%%%%%%%%%%%%%%%%%%%
\appendix{Second appendix title}
\noindent{\large\bf Put appropriate topic here} \\

Next, we show how $\mathrm{Var}\{X_n\}$ can be determined.  Let
$C_{\lambda}(l)$ be the autocovariance function of $\lambda_n$.  The
MVA technique basically approximates the input process $\lambda_n$
with a Gaussian process, which allows $\mathrm{Var}\{X_n\}$ to be
represented by the autocovariance function.  In particular, the
variance of $X_n$ can be expressed in terms of $C_{\lambda}(l)$ as
\begin{equation}
  \mathrm{Var}\{X_n\} = nC_{\lambda}(0) + 2\sum_{l=1}^{n-1} (n-l)C_{\lambda}(l)
\end{equation} 

\noindent{\large\bf Add more topic as you need} \\

Therefore, $C_{\lambda}(l)$ must be known in the MVA technique, either
by assuming specific traffic models or by off-line analysis in case of
traces.  In most practical situations, $C_{\lambda}(l)$ will not be
known in advance, and an on-line measurement algorithm developed
in~\cite{eun01} is required to jointly determine both $n^\ast$ and
$m_x$. For fGn traffic, $\mathrm{Var}\{X_n\}$ is equal to $\sigma^2
n^{2H}$, where $\sigma^2 = \mathrm{Var}\{\lambda_n\}$, and we can find
the $n^\ast$ that minimizes (\ref{eq:mx}) directly. Although $\lambda$
can be easily measured, it is not the case for $\sigma^2$ and $H$.
Consequently, the MVA technique suffers from the need of prior
knowledge traffic parameters. 





\end{document}
