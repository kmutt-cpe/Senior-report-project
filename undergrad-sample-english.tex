%%%%%% Run at command line, run
%%%%%% xelatex grad-sample.tex 
%%%%%% for a few times to generate the output pdf file
\documentclass[12pt,oneside,openright,a4paper]{cpe-english-project}

\usepackage{polyglossia}
\setdefaultlanguage{english}
\setotherlanguage{thai}
\newfontfamily\thaifont[Script=Thai,Scale=1.23]{TH Sarabun New}
\defaultfontfeatures{Mapping=tex-text,Scale=1.0,LetterSpace=0.0}
\setmainfont[Scale=1.0,LetterSpace=0,WordSpace=1.0,FakeStretch=1.0]{Times New Roman}
%\setmathfont(Digits)[Scale=1.0,LetterSpace=0,FakeStretch=1.0]{Times New Roman}


%%%%%%%%%%%%%%%%%%%%%%%%%%%%%%%%%%%%%%%%%%%%%%%%%%%%%%%%%%%%%%%%%%%
% Customize below to suit your needs 
% The ones that are optional can be left blank. 
%%%%%%%%%%%%%%%%%%%%%%%%%%%%%%%%%%%%%%%%%%%%%%%%%%%%%%%%%%%%%%%%%%%
% First line of title
\def\disstitleone{KMUTT CPE Chatbot System}   
% Second line of title
\def\disstitletwo{}   
% Your first name and lastname
\def\dissauthor{Mr. Nathaphop Sundarabhogin}   % 1st member
%%% Put other group member names here ..
\def\dissauthortwo{Ms. Natkanok Poksappaiboon}   % 2nd member (optional)
\def\dissauthorthree{Mr. Natthawat Tungruethaipak}   % 3rd member (optional)


% The degree that you're persuing..
\def\dissdegree{Bachelor of Engineering} % Name of the degree
\def\dissdegreeabrev{B.Eng} % Abbreviation of the degree
\def\dissyear{2020}                   % Year of submission
\def\thaidissyear{2563}               % Year of submission (B.E.)

%%%%%%%%%%%%%%%%%%%%%%%%%%%%%%%%%%%%%%%%%%%%
% Your project and independent study committee..
%%%%%%%%%%%%%%%%%%%%%%%%%%%%%%%%%%%%%%%%%%%%
\def\dissadvisor{Asst. Prof. Santitham Prom-on, Ph.D.}  % Advisor
%%% Leave it empty if you have no Co-advisor
\def\disscoadvisor{}  % Co-advisor
\def\disscommitteetwo{Assoc. Prof. Dr. Peerapon Siripongwutikorn, Ph.D.}  % 3rd committee member (optional)
\def\disscommitteethree{Asst. Prof. Nuttanart Facundes, Ph.D.}   % 4th committee member (optional) 
\def\disscommitteefour{Assoc. Prof. Thumrongrat Amornraksa, Ph.D.}    % 5th committee member (optional) 

\def\worktype{Project} %%  Project or Independent study
\def\disscredit{3}   %% 3 credits or 6 credits


\def\fieldofstudy{Computer Engineering} 
\def\department{Computer Engineering} 
\def\faculty{Engineering}

\def\thaifieldofstudy{วิศวกรรมคอมพิวเตอร์} 
\def\thaidepartment{วิศวกรรมคอมพิวเตอร์} 
\def\thaifaculty{วิศวกรรมศาสตร์}
 
\def\appendixnames{Appendix} % todo: Select Appendices or Appendix

\def\thaiworktype{ปริญญานิพนธ์} %  Project or research project % 
\def\thaidisstitleone{ระบบแชทบอทภาควิชาวิศวกรรมคอมพิวเตอร์ มจธ.}
\def\thaidisstitletwo{}
\def\thaidissauthor{นายณฐาภพ สุนทรโภคิน}
\def\thaidissauthortwo{นางสาวณัฐกนก โภคทรัพย์ไพบูลย์} %Optional
\def\thaidissauthorthree{นายณัฐวัฒน์ ตั้งฤทัยภักดิ์} %Optional

\def\thaidissadvisor{ผศ.ดร. สันติธรรม พรหมอ่อน}
%% Leave this empty if you have no co-advisor
\def\thaidisscoadvisor{} %Optional
\def\thaidissdegree{วิศวกรรมศาสตรบัณฑิต}

% Change the line spacing here...
\linespread{1.15}

%%%%%%%%%%%%%%%%%%%%%%%%%%%%%%%%%%%%%%%%%%%%%%%%%%%%%%%%%%%%%%%%
% End of personal customization.  Do not modify from this part 
% to \begin{document} unless you know what you are doing...
%%%%%%%%%%%%%%%%%%%%%%%%%%%%%%%%%%%%%%%%%%%%%%%%%%%%%%%%%%%%%%%%


%%%%%%%%%%%% Dissertation style %%%%%%%%%%%
%\linespread{1.6} % Double-spaced  
%%\oddsidemargin    0.5in
%%\evensidemargin   0.5in
%%%%%%%%%%%%%%%%%%%%%%%%%%%%%%%%%%%%%%%%%%%
%\renewcommand{\subfigtopskip}{10pt}
%\renewcommand{\subfigbottomskip}{-5pt} 
%\renewcommand{\subfigcapskip}{-6pt} %vertical space between caption
%                                    %and figure.
%\renewcommand{\subfigcapmargin}{0pt}

\renewcommand{\topfraction}{0.85}
\renewcommand{\textfraction}{0.1}

\newtheorem{theorem}{Theorem}
\newtheorem{lemma}{Lemma}
\newtheorem{corollary}{Corollary}

\def\QED{\mbox{\rule[0pt]{1.5ex}{1.5ex}}}
\def\proof{\noindent\hspace{2em}{\itshape Proof: }}
\def\endproof{\hspace*{\fill}~\QED\par\endtrivlist\unskip}
%\newenvironment{proof}{{\sc Proof:}}{~\hfill \blacksquare}
%% The hyperref package redefines the \appendix. This one 
%% is from the dissertation.cls
%\def\appendix#1{\iffirstappendix \appendixcover \firstappendixfalse \fi \chapter{#1}}
%\renewcommand{\arraystretch}{0.8}
%%%%%%%%%%%%%%%%%%%%%%%%%%%%%%%%%%%%%%%%%%%%%%%%%%%%%%%%%%%%%%%%
%%%%%%%%%%%%%%%%%%%%%%%%%%%%%%%%%%%%%%%%%%%%%%%%%%%%%%%%%%%%%%%%


\begin{document}
\pdfstringdefDisableCommands{%
\let\MakeUppercase\relax
}
\begin{center}
  \includegraphics[width=2.8cm]{./img/cover/logo02.jpg}
\end{center}
\vspace*{-1cm}

\maketitlepage
\makesignaturepage 

%%%%%%%%%%%%%%%%%%%%%%%%%%%%%%%%%%%%%%%%%%%%%%%%%%%%%%%%%%%%%%
%%%%%%%%%%%%%%%%%%%%%% English abstract %%%%%%%%%%%%%%%%%%%%%%%
%%%%%%%%%%%%%%%%%%%%%%%%%%%%%%%%%%%%%%%%%%%%%%%%%%%%%%%%%%%%%%
\abstract

Our world is becoming more technology driven. The population needs to have an awareness of technology 
like Artificial Intelligence (AI) for the country to become successful and reduce the workload of humans, as 
in the Computer Engineering department of King Mongkut’s University of Technology Thonburi. \\
Nowadays, many outsiders and insiders, whether students, teacher assistants, and staff have asked many 
questions about departments, such as frequently asked questions (FAQ) or the curriculum of each year etc. 
Each day, the staff in the department have to answer the same questions repeatedly. \\
Therefore, our team wants to develop the KMUTT CPE Chatbot System to help in answering the frequently 
asked questions and general inquiries of the Computer Engineering department in KMUTT to outsiders. It 
can also bring questions that chatbot cannot reply to officers and administration for answering the question 
and then the chatbot will send the answer back to the questioner via the Facebook Messenger platform.

\begin{flushleft}
\begin{tabular*}{\textwidth}{@{}lp{0.8\textwidth}}
\textbf{Keywords}: & Knowledge management / Chatbot / Artificial Intelligence (AI) / Siamese Neural Network / MySQL / Database / Automation / FAQ
\end{tabular*}
\end{flushleft}
\endabstract

%%%%%%%%%%%%%%%%%%%%%%%%%%%%%%%%%%%%%%%%%%%%%%%%%%%%%%%%%%%%%%
%%%%%%%%%% Thai abstract here %%%%%%%%%%%%%%%%%%%%%%%%%%%%%%%%%
%%%%%%%%%%%%%%%%%%%%%%%%%%%%%%%%%%%%%%%%%%%%%%%%%%%%%%%%%%%%%%
{
\XeTeXlinebreaklocale "th_TH"	
\thaifont
\thaiabstract
ในโลกของเราได้มีการขับเคลื่อนด้วยเทคโนโลยีที่มากขึ้น ดังนั้นประชากรจึงต้องมีการปรับตัวและรู้ทันถึงเทคโนโลยี เช่น ปัญญาประดิษฐ์ หรือ 
เอไอ เพื่อที่ประเทศนั้นจะสามารถพัฒนาและประสบความสำเร็จในการลดการใช้แรงงานของบุคลากรลง อนึ่งเช่นบุคลากรของภาควิชาวิศวกรรม
คอมพิวเตอร์ มหาวิทยาลัยเทคโนโลยีพระจอมเกล้าธนบุรี\\
ปัจจุบัน ทั้งคนภายนอกและภายใน ไม่แม้แต่นักศึกษา คุณครู ผู้ช่วยสอน หรือบุคลากรนั้น ล้วนแล้วแต่มีคำถามเกี่ยวกับภาควิชาวิศวกรรม
คอมพิวเตอร์ เช่น หลักสูตร การเรียนการสอนแต่ละปี ฯลฯ จนเกิดเป็นคำถามที่สามารถพบเห็นได้บ่อย ทำให้ภาควิชานั้นต้องมีการตอบ
คำถามที่บ่อยครั้งและเดิม ๆ\\
ดังนั้นทางผู้จัดทำจึงมีการเล็งเห็นปัญหาและต้องการที่จะทำโปรแกรมแชทบอทของภาควิชาวิศวกรรมคอมพิวเตอร์ มหาวิทยาลัยเทคโนโลยี
พระจอมเกล้าธนบุรีขึ้นมา เพื่อเป็นการช่วยตอบคำถามที่พบเห็นได้บ่อยและข้อสงสัยต่าง ๆ เกี่ยวกับภาควิชาแก่บุคคลภายนอกได้ โดยโปรแกรม
นั้นยังสามารถนำเอาคำถามที่แชทบอทไม่สามารถตอบได้นั้นส่งต่อไปให้แก่บุคลากรหรือเจ้าหน้าที่ดูแลเป็นผู้ตอบคำถามเหล่านั้นโดยผ่าน
\\แอปพลิเคชัน Facebook Messenger

\begin{flushleft}
\begin{tabular*}{\textwidth}{@{}lp{0.8\textwidth}}
 & \\

\textbf{คำสำคัญ}: & การจัดการความรู้ / Chatbot / ปัญญาประดิษฐ์ / Siamese Neural Network / MySQL / ฐานข้อมูล / Automation / คำถามที่พบบ่อย
\end{tabular*}
\end{flushleft}
\endabstract
}

%%%%%%%%%%%%%%%%%%%%%%%%%%%%%%%%%%%%%%%%%%%%%%%%%%%%%%%%%%%%
%%%%%%%%%%%%%%%%%%%%%%% Acknowledgments %%%%%%%%%%%%%%%%%%%%
%%%%%%%%%%%%%%%%%%%%%%%%%%%%%%%%%%%%%%%%%%%%%%%%%%%%%%%%%%%%
\preface
First, we would like to express our gratitude toward our advisor Asst. Prof. Santitham Prom-on whom always 
there to help us. When we need some advice, had a question about our research or a problem in writing, he 
always supports us even in times that he is busy, he still finds some time or alternative way to help us do our 
research. We are very grateful for what you have done for us, thank you.
\\Secondly, we owe our deep gratitude to all our committees: Assoc. Prof. Peerapon Siripongwutikorn, Asst. 
Prof. Nuttanart Facundes, and Assoc. Prof. Thumrongrat Amornraksa, who interest in our project and guided 
us all along until the completion.  By providing all the necessary information for developing a good system, 
give us advice and comment to make our project and ourselves to become better.
\\Last, we would like to thank our friends and family who, support us and help us in many ways besides the 
project, such as encouragement, guideline, and many other ways. We are very grateful toward all of them, 
thank you.


%%%%%%%%%%%%%%%%%%%%%%%%%%%%%%%%%%%%%%%%%%%%%%%%%%%%%%%%%%%%%
%%%%%%%%%%%%%%%% ToC, List of figures/tables %%%%%%%%%%%%%%%%
%%%%%%%%%%%%%%%%%%%%%%%%%%%%%%%%%%%%%%%%%%%%%%%%%%%%%%%%%%%%%
% The three commands below automatically generate the table 
% of content, list of tables and list of figures
\tableofcontents                    
\listoftables
\listoffigures                      

%%%%%%%%%%%%%%%%%%%%%%%%%%%%%%%%%%%%%%%%%%%%%%%%%%%%%%%%%%%%%%
%%%%%%%%%%%%%%%%%%%%% List of symbols page %%%%%%%%%%%%%%%%%%%
%%%%%%%%%%%%%%%%%%%%%%%%%%%%%%%%%%%%%%%%%%%%%%%%%%%%%%%%%%%%%%
% You have to add this manually..
\listofsymbols
\begin{flushleft}
\begin{tabular}{@{}p{0.07\textwidth}p{0.7\textwidth}p{0.1\textwidth}}
\textbf{SYMBOL}  & & \textbf{UNIT} \\[0.2cm]
$\alpha$ & Test variable\hfill & m$^2$ \\
$\lambda$ & Interarival rate\hfill &  jobs/second\\
$\mu$ & Service rate\hfill & jobs/second\\
\end{tabular}
\end{flushleft}
%%%%%%%%%%%%%%%%%%%%%%%%%%%%%%%%%%%%%%%%%%%%%%%%%%%%%%%%%%%%%%
%%%%%%%%%%%%%%%%%%%%% List of vocabs & terms %%%%%%%%%%%%%%%%%
%%%%%%%%%%%%%%%%%%%%%%%%%%%%%%%%%%%%%%%%%%%%%%%%%%%%%%%%%%%%%%
% You also have to add this manually..
\listofvocab
\begin{flushleft}
\begin{tabular}{@{}p{1in}@{=\extracolsep{0.5in}}l}
ABC & Adaptive Bandwidth Control \\
MANET & Mobile Ad Hoc Network 
\end{tabular}
\end{flushleft}

%\setlength{\parskip}{1.2mm}

%%%%%%%%%%%%%%%%%%%%%%%%%%%%%%%%%%%%%%%%%%%%%%%%%%%%%%%%%%%%%%%
%%%%%%%%%%%%%%%%%%%%%%%% Main body %%%%%%%%%%%%%%%%%%%%%%%%%%%%
%%%%%%%%%%%%%%%%%%%%%%%%%%%%%%%%%%%%%%%%%%%%%%%%%%%%%%%%%%%%%%%


\chapter{Introduction}

\section{Background} 

Explain the background of your works for readers. You can refer to figure by like this.. Figure~\ref{fig:x1}.
\begin{figure}[!h]
\caption{This is the figure x11}\label{fig:x1}
\end{figure}

Get ready, skanks! It's time for the truth train! Fame was like a drug. But what was even more like a drug were the drugs. Attempted murder? Now honestly, what is that? Do they give a Nobel Prize for attempted chemistry?

"Thank the Lord"? That sounded like a prayer. A prayer in a public school. God has no place within these walls, just like facts don't have a place within an organized religion. Thank you, steal again. I hope this has taught you kids a lesson: kids never learn.

…And the fluffy kitten played with that ball of string all through the night. On a lighter note, a Kwik-E-Mart clerk was brutally murdered last night. Oh, so they have Internet on computers now!

You don't like your job, you don't strike. You go in every day and do it really half-assed. That's the American way. Lisa, vampires are make-believe, like elves, gremlins, and Eskimos. Jesus must be spinning in his grave!

I prefer a vehicle that doesn't hurt Mother Earth. It's a go-cart, powered by my own sense of self-satisfaction. Marge, it takes two to lie. One to lie and one to listen. Attempted murder? Now honestly, what is that? Do they give a Nobel Prize for attempted chemistry?

I was saying "Boo-urns." Bart, with \$10,000 we'd be millionaires! We could buy all kinds of useful things like…love! I'll keep it short and sweet — Family. Religion. Friendship. These are the three demons you must slay if you wish to succeed in business.

How could you?! Haven't you learned anything from that guy who gives those sermons at church? Captain Whatshisname? We live in a society of laws! Why do you think I took you to all those Police Academy movies? For fun? Well, I didn't hear anybody laughing, did you? Except at that guy who made sound effects. Makes sound effects and laughs. Where was I? Oh yeah! Stay out of my booze. I can't go to juvie. They use guys like me as currency.

Oh, loneliness and cheeseburgers are a dangerous mix. "Thank the Lord"? That sounded like a prayer. A prayer in a public school. God has no place within these walls, just like facts don't have a place within an organized religion.


\section{Motivations}
Explain the motivations of your works.  
\begin{itemize}
\item   What are the problems you are addressing? 
\item  Why they are important?
\item  What are the limitations of existing approaches? 
\end{itemize}
You may combine this section with the background section.


\section{Problem Statements}

\section{Objectives}


\section{Scope of Work}

Explain the scope of your works. 
\begin{itemize}
\item   What are the problems you are addressing? 
\item  Why they are important?
\item  What are the limitations of existing approaches?
\end{itemize}

\section{Project Schedule}


%%%%%%%%%%%%%%%%%%%%%%%%%%%%%%%%%%%%%%%%%%%%%%%%%%%%%%%%%%%%
%%%%%%%%%%%%%%  Literature Review %%%%%%%%%%%%%%%%%%%%%%%%%%
%%%%%%%%%%%%%%%%%%%%%%%%%%%%%%%%%%%%%%%%%%%%%%%%%%%%%%%%%%%%
\chapter{Background Theory and Related Work}

\section{Introduction}
This chapter, we are going to demonstrate the theory, core concept, tools, and related research to be background 
and terminology before deep down into methodology and result.

\section{Theory and Core Concepts}
\subsection{Knowledge Management}
Knowledge Management is the process of creating, sharing, gathering, using and managing the knowledge 
and information in organization (Koenig, 2018). It refers to a multidisciplinary approach to achieve organizational 
objectives by making the best use of knowledge.\\
In terms of our project, knowledge management is the website to keep the knowledge and information about 
computer engineering in question and answer format. Questions in knowledge management can group in category 
such as basic information, curriculum, frequently asked questions (FAQ), and registration. Also, in each category 
has subcategory to be more specific the question. Only staff in Computer Engineering department can use the knowledge 
management.
 
\subsection{Natural Language Processing}
Natural language processing (NLP) is a subfield of linguistics, computer science, and artificial
intelligence concerned with computers and human language interactions, mainly how to program
computers to process and analyze large amounts of natural language data.\\
The reason for the development of NLP was that computers were originally designed to be more suited
to understand numeric information or code (VENTURES, 2020). Which does not match the way of human
communication which relies on language and language is significantly more complex than computer-based
code. So, NLP emerged to bridge the gap in human-computer communication.\\
NLP supports both reading and listening by using other technologies such as Visual Recognition. For
reading text and using voice recognition for listening Including other technologies to show results
to humans Including being able to convey information back to us in both text and voice form as well.\\
NLP currently has a 6-step language learning process:
\begin{enumerate}
  \item Morphological Level — Understand letters. NLP will remove words into messages, find consonants,
  vowels, and spelling for the next step.
  \item Lexical Level — Understand the word after mixing the letters. It will start to find the meaning
  of that word to prepare for understanding the whole sentence.
  \item Syntactic Level — Understand sentences. Based on understanding the common word and structural
  order specified by experts or schemes learned.
  \item Semantic Level — Understand the context of words in a sentence. Understand the meaning of terms
  used in sentences that are outside the standard language structure.
  \item Discourse Level — Understand the syntax. Understand the effect of the previous sentence on the
  meaning of the sentence that being read. This includes understanding the order of words used in sentences
  that give different meanings.
  \item Pragmatic Level — Understand the meaning of words and sentences based on the actual situation
  or knowledge base, which may not be specified in the content. So, it can be interpreted as close to
  human beings that can always relate the new information with the old knowledge. In addition to
  understanding each point, NLP has three other language learning channels modeled after human language learning:
  \begin{itemize}
    \item Symbolic  — Symbolic is the basis of human language understanding where AI has to understand vocabulary up
    to that language's structure. Developers can now apply expert knowledge directly into AI.
    \item Statistical — After learning the basics of the language, the next step will be to collect information
    on using the language in different places. The patterns were analyzed by statistical methods such as the
    frequency of words used. Find out how to sort common sentences. Then bring to synthesize new knowledge.
    This will help AI improve the language based on its current popularity. And understand the use of language
    in a particular field such as science, finance, or various academic documents, etc.
    \item Connectionist AI — Connectionist AI combines a statistical language learning process with a symbolic level for complete
    communication and understanding. This is based on the Symbolic stage's actual knowledge and adapted with new
    information received from the statistical stage.
    
  \end{itemize}
\end{enumerate}

\subsection{Chatbot}
Chatbot is an artificial intelligence software used to communicate by talking to humans (Team, 2020).
For the benefit of one way or another through messaging applications, websites, mobile apps or through
the telephone. Creating or using chatbot is fun, can be used for counselling or contact interface.\\
Chatbot might have cons as
\begin{itemize}
  \item The chatbot itself does not really understand the conversation
  \item Scripting the chatbot with rules is fragile and hard to be fully written
  \item Conversation quality is limited to the data.
\end{itemize}
There are two type of chatbot are IR based Chatbot and Neural Network Based Chatbot % todo: (Team, 2020; Savina, 2019).
\subsubsection{IR Based Chatbot}
IR based chatbot uses collections of conversations in order to produce responses to conversation %todo; (Savina, 2019; flow.ai, n.d.).
Collections of conversations are stored in sequence of turns. When the chatbot receives a text,
the chatbot can produce its turn of conversation by
/\begin{itemize}
  \item Using the received text to search for a turn with the closest match in the collection,
  then return the next turn as response (most initiative).
  \item Using the received text to search for a turn with the closest match in the collection,
  then return that matching turn as response (not initiative, but actually provide better result).
\end{itemize}
The turn searching is not limited to the receiving text, but rather the whole conversation so far
can be used for response searching. Downside of this approach is that the text that chatbot produces
might not be consistent. Chatbot might say something that conflicts with what the chatbot has said
before.
\subsubsection{Neural Network Based Chatbot}
Neural network based chatbot sees its response as a transformation of the incoming text. This chatbot
provides responses by training an encoder-decoder neural network with pairs of turns\\ %todo: (Savina, 2019; flow.ai, n.d.).
An encoder-decoder neural network has an encoder and a decoder. The encoder transforms incoming
text into some abstract representation, and the decoder transforms that abstract representation
into response text. The encoder and the decoder usually have RNN structure, as they are capable
of handling variable length of input text and return variable length response text.

\subsection{Artificial Intelligence (AI)}
AI is a computing system that provides deep analysis like human intelligence and can produce
actionable results. For example, translation is due to the processing of incoming messages
and convert it into another language and so on.\\
AI learning is not different from human learning which “remember” and “think” like a human.
For example, children who see their parents' faces repeatedly every day and try to know whose
person is “dad” or “mother” for a long time. So children will be able to look at their face and
call “Dad” or “Mom” correctly.\\
The stimulus used to train AI is “data”, which teaches children to call “parents” must be trained
many times and must use information that has the same repetitive characteristics.\\ %todo: (Coraline, n.d.).
The AI mechanism is a computer processing system, and Machine learning is a component in AI.
This machine learning has a variety of algorithms depending on the problem. Most data that used to
trend such as deep learning is an algorithm that is suitable for large complex data or Random Forests
is an algorithm that used for supervised problems, etc.\\
In summary, Artificial Intelligence (AI) is like a tool, a robot, or whatever the action takes place.
For example a smart vacuum cleaner with sensors to prevent stairs fall and detect unmanned dust
vehicles. Or even the Facebook that we are using, we can suggest new media that the customer is
interested. Let us learn from what we are interested in. The brain of AI is machine learning (ML),
learning from what we stimulate and the output is a number or code that is forwarded to display
the result or allow the AI to show action. ML can be used in various ways, from credit scoring
to analyzing a person's financial credibility, weather forecast sound analysis, or even the
translation of the language that ML is learned. It requires a programmatic mechanism or a variety
of algorithms, with data scientist designing is one of the most popular algorithms.\\
Deep learning is intended to be easy to use. However, in real work, data scientist needs to
design variables in the deep learning and need to find other algorithms as a comparison pair
to look for the most suitable algorithm in actual use.

\subsection{Deep Neural network}
Deep learning algorithms must use Artificial Neural Networks (ANN), which is the same as how
the nervous system works in the human brain (Coraline, n.d.). These networks have neurons that
connect from the nervous system and communicate. It uses parallel processing to understand and
learn from the vast amount of information it continually receives.\\
The general principle of deep learning is to have multiple processors. The input data for each
layer is obtained through interaction with the other layers. Deep learning seeks to find deeper
relationships as the number of layers and processors in it increases. The higher the data, the
deeper and more complex (abstract).\\
The deep learning structure architecture is based on the greedy method to find things in each
layer that make deep learning more efficient than other methods. For example, early information
might learn that an incoming image consists of lines. The high floor brings together the lines
to form a rectangle. And the next layer is to find the correlation of the squares until the
computer recognizes that the image is the image of flag, etc.

\subsection{Recurrent Neural network (RNN)}
RNN is a class of artificial neural networks, but the difference is there is a connection between
each state to form a memory called “Internal memory” %todo: (Shekhar, 2019)
. Unlike the feedforward neural network, RNN can use the internal memory to process the input that is sequential data.

\begin{figure}[!h] \centering
  \setlength{\fboxrule}{0.2mm} % line border
  \setlength{\fboxsep}{0.5cm} % space between picture and line border
  \fbox{\includegraphics[width=14cm]{./img/ch3/example_rnn.png}}  %width=14cm equal to width of paragraph
  \caption{An example of RNN} % Caption under figure
  \label{fig:example_rnn} % Used for ref in paragraph
\end{figure}

Figure \ref*{fig:example_rnn} the method that RNN uses to create the internal memory is
creating a loop within it. Figure A is activated, gets the input $X_t$ in the state of time,
and output value to the hidden layer. The loop allows the information to be passed into the
next state. Therefore, the first input will affect the output in the last state.

\begin{figure}[!h] \centering
  \setlength{\fboxrule}{0.2mm} % line border
  \setlength{\fboxsep}{0.5cm} % space between picture and line border
  \fbox{\includegraphics[width=14cm]{./img/ch3/standard_rnn.png}}  %width=14cm equal to width of paragraph
  \caption{A standard RNN method} % Caption under figure
  \label{fig:standard_rnn} % Used for ref in paragraph
\end{figure}

Inside the RNN layer, the methods are different for each type as you can see from Figure
\ref*{fig:standard_rnn}, standard RNN method. It takes the output value from the previous
state and then combines with the present state’s input. After combining the result will go
to the nonlinear function, in this case is Tanh. Then the output from this state will be
used in the later state. The figure above can transform into the formula below.

\[h_t = \Phi(Wx_t + Uh_{t-1})\]
\begin{align*}
&h(t) = \text{A hidden layer or output value of this state.}\\
&\Phi = \text{An activation function, either sigmoid function or tanh.}\\
&W = \text{The weight of the input.}\\
&x(t) = \text{A present state input.}\\
&U = \text{A hidden state matrix.}\\
&h(t-1) = \text{A hidden layer or output value of the last state.}
\end{align*}

Because of this method, many consider RNN to use whenever they need context from the past,
such as speech recognition, video analysis, and music composition.

\subsection{Long Short-Term Memory (LSTM)}
Like RNN the Long Short-Term Memory network is also a recurrent neural network but with
different architecture designs %todo:(Shekhar, 2019; Olah, 2015)
, as seen in Figure \ref*{fig:lstm_arch_design}. LSTM is specifically used to avoid the long-term dependency problem
or the vanishing gradient.

\begin{figure}[!h] \centering
  \setlength{\fboxrule}{0.2mm}
  \setlength{\fboxsep}{0.5cm}
  \fbox{\includegraphics[width=14cm]{./img/ch3/lstm_arch_design.jpg}}
  \caption{LSTM architecture design}
  \label{fig:lstm_arch_design}
\end{figure}

The core idea of LSTM is the cell state, the horizontal line that goes between each state. The
cell state is like a bridge to carry out information from the first state throughout the model.
Besides the ability to carry information, LSTM can also manipulate the cell state by adding gates
that interact with the cell state. Each gate can alter the information on cell state like
forgetting some data or to determine which information needs to be remembered. 

\subsection{MaLSTM similarity function}
\[\exp(-\|h^{(left)} - h^{(right)}\|1)\]
In MaLSTM the identical sub-network is the way from the embedding up to the last LSTM hidden
state (Primo.ai, 2020; Brown, 2020). using a LSTM to read in word-vectors that represent each
input sentence and employs final hidden state as a vector representation for each sentence.
The similarities among these representations are employed as predictors of semantic similarity.

\subsection{The architecture of Siamese Neural Network}

\begin{figure}[!h]
  \centering
  \setlength{\fboxrule}{0.2mm}
  \setlength{\fboxsep}{0.5cm}
  \fbox{\includegraphics[width=14cm]{./img/ch3/siamese_arch.png}}
  \caption{Sample Siamese Neural Networks architecture}
  \label{fig:siamese_arch}
\end{figure}
Siamese Neural Networks is a type of neural network that contains multiple instances of the
same model and shares the same architecture and weights (twin networks)%todo:(Brown, 2020; Cohen, 2017; J, 2020)
. This architecture shows its strength when it must learn with limited data, and we do not have a complete dataset,
for example Zero or One-shot learning tasks. They work in parallel and are responsible for creating vector
representations for the input and producing better vector representations by measuring similarities between vectors. 

\section{Tools}
Figure \ref*{fig:ch3_tools} is an overview of tools that are used in KMUTT CPE Chatbot in each subsystem.
\begin{figure}[h!]
  \centering
  \setlength{\fboxrule}{0.2mm}
  \setlength{\fboxsep}{0.5cm}
  \fbox{\includegraphics[width=14cm]{./img/ch3/tools.png}}
  \caption{All tools that are used in KMUTT CPE Chatbot System.}
  \label{fig:ch3_tools}
\end{figure}

\subsection{Knowledge Management Subsystem}
\subsubsection{React JS}
React is a JavaScript library, or it can be called a JavaScript Framework that we use for building
our web pages to look interesting. The strength of React that makes it worth using is that It has
a built-in caching system that makes our web pages responsive. It is very suitable for SPA
implementation. Writing React, we can also separate the elements of our web page into parts,
called as a component, and assemble it into a web page. This allows us to reuse our components
elsewhere %todo:(Programmer, 2019)
. Do not waste time rewriting.

\subsubsection{GraphQL}
GraphQL is a language for accessing data (Query Language) for the use of APIs of the system %todo: (Athiwat, 2017)
and will execute commands on the server side, also known as server-side runtime, using our defined
data structures.

\subsection{Prediction Model Subsystem}
\subsubsection{Keras}
Keras is an open-source neural network library written in python. It uses Tensorflow as one of the
backends using both CPU and GPU %todo:(Keras: the Python deep learning API, 2015)
. It is designed for a fast experiment with Deep Neural Network. It was developed as part of
the research effort of project ONEIROS (Open-ended Neuro-Electronic Intelligent Robot Operating
System). Keras library is commonly used to build neural networks such as network layers,
activation functions, optimizers, etc.

\subsubsection{TensorFlow}
Tensorflow is an open-source library for numerical computation and large-scale machine learning.
It uses Python to provide a convenient front-end API for building applications with the framework
while executing those applications in C++ %todo:(Yegulalp, 2019)
.  TensorFlow can train and run deep neural networks for handwritten digit classification,
image recognition, word embeddings, recurrent neural networks, and is the backend for other
libraries.

\subsubsection{PyThaiNLP}
PyThaiNLP is a library package of Python languages used to process text. And language analysis,
similar to NLTK, but especially for the Thai language %todo: (Surapong, 2020)
. There are a variety of functions such as character set, Thai alphabet, Thai words,
stop words in Thai, cut Thai words, Type of grammatical words, spell check, correct words,
and much more.

\subsubsection{Apache Airflow}
Apache Airflow is an Open Source that manages various tasks by writing a Configuration in
the Python Code, which is suitable for Python programmers. Each task can see the workflow
in detail. When there is a problem, such as a bottleneck, it can be analyzed easily %todo: (Apache Airflow, 2019)
. We can set the working time like a Cronjob. For example, when running Job, A at 1:00 am
every day, Run Job B at 8:00 am, once a week on Sunday. Job is called DAGs
(Directed Acyclic Graphs) in Airflow is a web UI for us to monitor Task Failure
that occurred, or Duration Time caused by each work.

\subsection{Chatbot System}
\subsubsection{Dialogflow}
Dialogflow is a chatbot builder from Google which provide flexible integration to many platforms,
unique to Natural Language Processing, or NLP %todo: (Rouse, 2019)
which means that chatbot can accurately understand the meaning of user-typed sentences,
enabling chatbot to interact with people. Use it precisely and to the point.

\subsubsection{Facebook Messenger}
Facebook Messenger is an application through a smartphone and desktop PC. It connects information
and information, extending from the inbox system, sending messages on Facebook, and creating data
storage to facilitate communication %todo:(Facebook Messenger, 2020)
.\\
We decide this application to be our chat platform because from our survey (Figure \ref*{fig:ch3_result_survey})
through insiders and outsiders, they prefer to use the Facebook Messenger platform the most.

\begin{figure}[h!]
  \centering
  \setlength{\fboxrule}{0.2mm}
  \setlength{\fboxsep}{0.5cm}
  \fbox{\includegraphics[width=14cm]{./img/ch3/result_survey.png}}
  \caption{The result of our survey from 108 people.}
  \label{fig:ch3_result_survey}
\end{figure}
\subsection{ Storage Subsystem}
\subsubsection{NestJS}
NestJS is a framework for building Node.js server-side applications that scalable and very
efficient %todo: (Mateusz, 2018)
. NestJS uses progressive JavaScript which means NestJS updated
features inside follow latest JavaScript. In addition, NestJS is very flexible from giving
a freedom to easily plug the third party or another module to NestJS application. This reason
makes NestJS extensible and can be scalable to big size project.\\
Moreover, NestJS supports TypeScript and comes up with idea combinantion of Object-Oriented
Programming (OOP), Functional Programming (FP), and Functional Reactive Programming (FRP)
to makes more friendly to develop the application %todo: (Mateusz, 2018)
. Also, NestJS provides a level of abstraction above common Node.js framework like Express or
Fastify by adding more level such as controller, provider, module, etc.

\subsubsection{MySQL}
It is a popular program used to manage database systems nowadays because of its free instruction
set. MySQL is a relational database (RDBMS: Relational Database Management System) that can
simultaneously work with multiple tables %todo: (Herawan, 2020)
. Those tables with shared fields MySQL are a capable database server. Standard database language
like ANSI SQL (Structured Queries Language).

\subsubsection{TypeORM}
TypeORM is Object Relational Mapping (ORM), which facilitates and simplifies database connection %todo:(TypeORM, 2019)
. We chose TypeORM because it supports TypeScript, which is suitable for NestJS.

\subsubsection{Bull}
Bull is a Node library that manages a queue system based on Redis, which benefits from the queue %todo: (NUIISAN, 2019)
. It can solve many problems elegantly, from smoothing out processing peaks to creating
robust communication channels between microservices or offloading heavy work from one server to
many smaller workers, etc.

\subsection{Deployment System}
\subsubsection{Docker}
Docker is an engine that works in a simulated environment on the server to run the required service.
It works similarly to Virtual Machines such as VMWare, VirtualBox, XEN, KVM, but the main difference
is Virtual %todo:(What is Docker, n.d.)
. Previously known machine It simulates the entire OS for use.

\subsubsection{Kubernetes}
Kubernetes, or k8s is an open-source platform that allows operations to Linux Containers
can be done automatically. Minimize the process of installing or extending applications
running on containers that developers have to do manually %todo: (Casey, 2020)
. Or it can be said that This enables developers to cluster a cluster of hosts
running Linux containers, and Kubernetes can quickly and efficiently manage them
for public, private, and hybrid cloud deployments.

\subsection{Languages}
\subsubsection{TypeScript}
TypeScript is a programming language that combines the capabilities that ES2015 itself has.
What is added is support for Type System and additional features such as Enum and the expanded
capabilities of object-oriented programming. TypeScript is a Babel-like transpiler %todo: (Tate, 2015)
. This means that the TypeScript translator will translate the code we write into JavaScript one more
time, ensuring that the result will be usable in a regular web browser.

\subsubsection{HTML}
HTML is the primary language used for writing web pages using Tag to define their display.
HTML stands for Hypertext Markup Language, where Hypertext refers to hyperlinked text.
Markup language refers to the language in which a tag is used.
HTML is defined as a language in which a tag is used to designate a web page connected
in hyperspace via Hyperlink %todo: (Anon, 2019)
. It is now developed and standardized by the World Wide Web Consortium (W3C).

\subsubsection{CSS}
CSS is a language used to decorate HTML / XHTML documents with appearance, colors, spacing,
backgrounds, borders, and more %todo: (Morris, 2020)
. As requested, CSS stands for Cascading Style Sheets. It is a syntax-specific programming
language standardized by W3C as one of the web site decoration languages. It has been widely
popular.
%%%%%%%%%%%%%%%%%%%%%%%%%%%%%%%%%%%%%%%%%%%%%%%%%%%%%
%%%%%%%%%%%%%%%%%%%%%%%%%%%%%%%%%%%%%%%%%%%%%%%%%%%%%
\chapter{Proposed Work}

Explain the design (how you plan to implement your work) of your project. Adjust the section titles below to suit the types of your work. Detailed physical design like circuits and source codes should be placed in the appendix.

\section{System Architecture}

\begin{table}[!h]
\centering
\caption{test table x1}\label{tbl:symbols}
\begin{tabular}{@{}p{0.07\textwidth}|p{0.7\textwidth}p{0.1\textwidth}}\hline
\multicolumn{2}{l}{\textbf{SYMBOL}}  & \textbf{UNIT} \\ \hline 
$\alpha$ & Test variable\hfill & m$^2$ \\
$\lambda$ & Interarrival rate\hfill &  jobs/second\\
$\mu$ & Service rate\hfill & jobs/second \\ \hline
\end{tabular}
%\begin{tabular}{c|c} \hline
% $\alpha$ & $\beta$ \\ \hline
% $\delta$ & $\mu$ \\ \hline
%\end{tabular}
\end{table}

\section{System Specifications and Requirements}

\section{Hardware Module 1}
\subsection{Component 1}
\subsection{Logical Circuit Diagram}

\section{Hardware Module 2}
\subsection{Component 1}
\subsection{Component 2}

\section{Path Finding Algorithm}

\section{Database Design}

\section{GUI Design}



%%%%%%%%%%%%%%%%%%%%%%%%%%%%%%%%%%%%%%%%%%%%%%%%%%%%%%%%%%%%%%
%%%%%%%%%%%%%%%%%%%% Experiments %%%%%%%%%%%%%%%%%%%%%%%%%%%%%
%%%%%%%%%%%%%%%%%%%%%%%%%%%%%%%%%%%%%%%%%%%%%%%%%%%%%%%%%%%%%%%
\chapter{Implementation Results}
\section{Chatbot Subsystem and Storage Subsystem Result}
Now our chatbot subsystem which we implement in dialog flow can greeting and understand what is the question and for storage subsystem which we implement in GraphQL query category from backend API. Now our system can category of FAQs by ID and can query FAQs from all category.

\begin{figure}[!h]\centering
\fbox{\includegraphics[width=15cm]{./img/ch4/ch4_1.png}}
\caption{Classification greeting messages and questions}\label{fig:Classification greeting messages and questions}
\end{figure}
\begin{figure}[!h]\centering
\fbox{\includegraphics[width= 15cm]{./img/ch4/ch4_2.png}}
\caption{GraphQL query FAQs category from backend API}\label{fig:GraphQL query FAQs category from backend API}
\end{figure}

\section{Model Prediction Subsystem Result}
\subsection{Training Siamese Neural Network with MaLSTM Result}
After preprocessing the data, we split the data into train, validate, and test data. Then separate the data into left inputs and right inputs, one for each side of MaLSTM. Then train the model with 100 epochs and use batch sizes 128. In addition to using early stops to prevent the model too overfitting and model checkpoint to save the model. Even our group set 100 epochs, but the model was trained only 17 epochs before stopping from early stop. Loss of training data is 0.4116 and loss of validation data is 0.4358. Also, accuracy of training data is 0.5487 and accuracy of validation data is 0.4358. Graphs below are the result from training.

\begin{figure}[!h]\centering
\fbox{\includegraphics[width= 15cm]{./img/ch4/ch4_3.png}}
\caption{Model accuracy}\label{fig:Model accuracy}
\end{figure}
\begin{figure}[!h]\centering
\fbox{\includegraphics[width= 15cm]{./img/ch4/ch4_4.png}}
\caption{Model loss}\label{fig:Model loss}
\end{figure}

\subsection{Evaluation Model Result}
The result from the model is the real number from 0 to 1. We interpret both questions are similar (result = 1) if the model predicted score is more than 0.3. If the model predicted score is less than 0.3, both questions are not similar (result = 0). We pick 0.3 to be a criterion for similar or not similar because we have picked 0.6 and 0.5 already but accuracy is very low. After the model predicts the test data, we convert from a similar score (0 to 1)  to binary (0 or 1) instead before evaluating. As you can see in the training topic, we use accuracy and loss to be the main criteria to evaluate the model while training. But in this process we will also use accuracy, precision, and F1-score to evaluate the model. The picture below is the result after evaluating the model with test data.

\begin{figure}[!h]\centering
\fbox{\includegraphics[width= 15cm]{./img/ch4/ch4_5.png}}
\caption{Classification report}\label{fig:Classification report}
\end{figure}
\begin{figure}[!h]\centering
\fbox{\includegraphics[width= 15cm]{./img/ch4/ch4_6.png}}
\caption{Confusion matrix}\label{fig:Confusion matrix}
\end{figure}

The accuracy after evaluating with test data is 0.527. From the confusion matrix and classification report, the overall model performs accuracy with 52.7\%. There are two classes and we got class o with less precision around 25\%. but we have got class 1 with high precision with 76\%.
\subsection{Siamese Neural Network with MaLSTM Model Result}

The picture below shows an example result from playing with the model. The Input questions are
{
\XeTeXlinebreaklocale "th_TH"	
\thaifont 
'วิศวะคอมเรียนหลักสูตรอะไรบ้างคะ' }
and
{
\XeTeXlinebreaklocale "th_TH"	
\thaifont 
 ' วิศวะคอมมีหลักสูตรอะไรบ้าง'. 
}Both questions are similar and our model can predict both questions are similar for 83.33\%.

\begin{figure}[!h]\centering
\fbox{\includegraphics[width= 15cm]{./img/ch4/ch4_7.png}}
\caption{Model result by compare questions which similar}\label{fig:Model result by compare questions which similar}
\end{figure}

From others example from playing with model.The Input questions are
{
\XeTeXlinebreaklocale "th_TH"	
\thaifont 
 ‘ วิศวะคอมเรียนเกี่ยวกับอะไรบ้าง’ }and 
{
\XeTeXlinebreaklocale "th_TH"	
\thaifont ‘ วิศวะคอมมีเรียนหลักสูตรอะไรบ้าง’.}
 Both questions are not similar and our model can predict both questions are not similar for 57.14\%

\begin{figure}[!h]\centering
\fbox{\includegraphics[width= 15cm]{./img/ch4/ch4_8.png}}
\caption{Model result by compare questions which not similar}\label{fig:Model result by compare questions which not similar}
\end{figure}

Even the model can predict, but the overall performance of the model is quite not good. Because we think the model should return similar questions from MaLSTM more than 0.5 scores not 0.3 scores. We think that the problem comes from not enough data. If we can gather more dataset and have more cases about question pairs that are similar or not similar, especially not similar that we got less precision, we can increase our model's performance. Also we have to integrate all subsystem to be one system in the production environment.

%%%%%%%%%%%%%%%%%%%%%%%%%%%%%%%%%%%%%%%%%%%%%%%%%%%%%%%%%%%%%%%
%%%%%%%%%%%%%%%%%%%% Conclusions %%%%%%%%%%%%%%%%%%%%%%%%%%%%%
%%%%%%%%%%%%%%%%%%%%%%%%%%%%%%%%%%%%%%%%%%%%%%%%%%%%%%%%%%%%%%%
\chapter{Conclusions}

This chapter is optional for proposal and progress reports but 
is required for the final report.

\section{Problems and Solutions}
State your problems and how you fixed them.

\section{Future Works}
What could be done in the future to make your projects better.

%%%%%%%%%%%%%%%%%%%%%%%%%%%%%%%%%%%%%%%%%%%%%%%%%%%%%%%%%%%%%%%
%%%%%%%%%%%%%%%%%%%% Bibliography %%%%%%%%%%%%%%%%%%%%%%%%%%%%%
%%%%%%%%%%%%%%%%%%%%%%%%%%%%%%%%%%%%%%%%%%%%%%%%%%%%%%%%%%%%%%%

%%%% Comment this in your report to show only references you have
%%%% cited. Otherwise, all the references below will be shown.
\nocite{*}
%% Use the kmutt.bst for bibtex bibliography style 
%% You must have cpe.bib and string.bib in your current directory.
%% You may go to file .bbl to manually edit the bib items.
\bibliographystyle{kmutt}
\bibliography{string,cpe}

%%%%%%%%%%%%%%%%%%%%%%%%%%%%%%%%%%%%%%%%%%%%%%%%%%%%%%%%%%%%%%%
%%%%%%%%%%%%%%%%%%%%%%%% Appendix %%%%%%%%%%%%%%%%%%%%%%%%%%%%%
%%%%%%%%%%%%%%%%%%%%%%%%%%%%%%%%%%%%%%%%%%%%%%%%%%%%%%%%%%%%%%%
\appendix{First appendix title}
\noindent{\large\bf Put appropriate topic here} \\

This is where you put hardware circuit diagrams, detailed experimental data in tables or source codes, etc.. \\ \bigskip



This appendix describes two static allocation methods for fGn (or fBm)
traffic. Here, $\lambda$ and $C$ are respectively the traffic arrival
rate and the service rate per dimensionless time step. Their unit are
converted to a physical time unit by multiplying the step size
$\Delta$. For a fBm self-similar traffic source,
Norros~\cite{norros95} provides its EB as
\begin{equation}\label{eq:norros}
  C = \lambda + (\kappa(H)\sqrt{-2\ln\epsilon})^{1/H}a^{1/(2H)}x^{-(1-H)/H}\lambda^{1/(2H)}
\end{equation}
where $\kappa(H) = H^H(1-H)^{(1-H)}$. Simplicity in the calculation is
the attractive feature of (\ref{eq:norros}).

The MVA technique developed in~\cite{kim01} so far provides the most
accurate estimation of the loss probability compared to previous
bandwidth allocation techniques according to simulation results.
Consider a discrete-time queueing system with constant service rate
$C$ and input process $\lambda_n$ with $\mathbb{E}\{\lambda_n\} =
\lambda$ and $\mathrm{Var}\{\lambda_n\} = \sigma^2$.  Define $X_n \equiv
\sum_{k=1}^n \lambda_k - Cn$.  The loss probability due to the MVA
approach is given by
\begin{equation}\label{eq:loss_mva}
  \varepsilon \approx \alpha e^{-m_x/2}
\end{equation}
where
\begin{equation}\label{eq:mx}
m_x = \min_{n \geq 0} \frac{((C-\lambda)n + B)^2}{\mathrm{Var}\{X_n\}} =
\frac{((C-\lambda)n^\ast + B)^2}{\mathrm{Var}\{X_{n^{\ast}}\}}
\end{equation} 
and 
\begin{equation}\label{eq:alpha}
  \alpha =
  \frac{1}{\lambda\sqrt{2\pi\sigma^2}}\exp\left(\frac{(C-\lambda)^2}{2\sigma^2}\right)
  \int_C^\infty (r-C)\exp\left(\frac{(r-\lambda)^2}{2\sigma^2}\right)\, dr
\end{equation}
For a given $\varepsilon$, we numerically solve for $C$ that satisfies
(\ref{eq:loss_mva}). Any search algorithm can be used to do the task.
Here, the bisection method is used.  

Next, we show how $\mathrm{Var}\{X_n\}$ can be determined.  Let
$C_{\lambda}(l)$ be the autocovariance function of $\lambda_n$.  The
MVA technique basically approximates the input process $\lambda_n$
with a Gaussian process, which allows $\mathrm{Var}\{X_n\}$ to be
represented by the autocovariance function.  In particular, the
variance of $X_n$ can be expressed in terms of $C_{\lambda}(l)$ as
\begin{equation}
  \mathrm{Var}\{X_n\} = nC_{\lambda}(0) + 2\sum_{l=1}^{n-1} (n-l)C_{\lambda}(l)
\end{equation} 
Therefore, $C_{\lambda}(l)$ must be known in the MVA technique, either
by assuming specific traffic models or by off-line analysis in case of
traces.  In most practical situations, $C_{\lambda}(l)$ will not be
known in advance, and an on-line measurement algorithm developed
in~\cite{eun01} is required to jointly determine both $n^\ast$ and
$m_x$. For fGn traffic, $\mathrm{Var}\{X_n\}$ is equal to $\sigma^2
n^{2H}$, where $\sigma^2 = \mathrm{Var}\{\lambda_n\}$, and we can find
the $n^\ast$ that minimizes (\ref{eq:mx}) directly. Although $\lambda$
can be easily measured, it is not the case for $\sigma^2$ and $H$.
Consequently, the MVA technique suffers from the need of prior
knowledge traffic parameters.


%%%%%%%%%%%%%%%%%%%%%%%%%%%%%%%%%%%%%%%%%%%%%%%%%%%%%%%%%%
%%%%%%%%%%%%%%% The 2nd appendix %%%%%%%%%%%%%%%%%%%%%%%%%%
%%%%%%%%%%%%%%%%%%%%%%%%%%%%%%%%%%%%%%%%%%%%%%%%%%%%%%%%%%
\appendix{Second appendix title}
\noindent{\large\bf Put appropriate topic here} \\

Next, we show how $\mathrm{Var}\{X_n\}$ can be determined.  Let
$C_{\lambda}(l)$ be the autocovariance function of $\lambda_n$.  The
MVA technique basically approximates the input process $\lambda_n$
with a Gaussian process, which allows $\mathrm{Var}\{X_n\}$ to be
represented by the autocovariance function.  In particular, the
variance of $X_n$ can be expressed in terms of $C_{\lambda}(l)$ as
\begin{equation}
  \mathrm{Var}\{X_n\} = nC_{\lambda}(0) + 2\sum_{l=1}^{n-1} (n-l)C_{\lambda}(l)
\end{equation} 

\noindent{\large\bf Add more topic as you need} \\

Therefore, $C_{\lambda}(l)$ must be known in the MVA technique, either
by assuming specific traffic models or by off-line analysis in case of
traces.  In most practical situations, $C_{\lambda}(l)$ will not be
known in advance, and an on-line measurement algorithm developed
in~\cite{eun01} is required to jointly determine both $n^\ast$ and
$m_x$. For fGn traffic, $\mathrm{Var}\{X_n\}$ is equal to $\sigma^2
n^{2H}$, where $\sigma^2 = \mathrm{Var}\{\lambda_n\}$, and we can find
the $n^\ast$ that minimizes (\ref{eq:mx}) directly. Although $\lambda$
can be easily measured, it is not the case for $\sigma^2$ and $H$.
Consequently, the MVA technique suffers from the need of prior
knowledge traffic parameters. 





\end{document}
